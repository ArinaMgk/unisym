\labex{ASM}

\section{Directive}

\subsection{Base Compatible Platform}

Usually, this is set by the category of the respective processor.
There are examples for Intel$^{\textregistered}$80386.

\B{for NASM}

\T{CPU 386} or \T{[CPU 386]}

\B{for MASM}

\T{.386}


\subsection{Calling and Memory Method}

\B{for MASM}

\verb`.model flat, stdcall` 

flat: a 4GB segment is shared for code and data.

\section{Instruction}
\B{Header}: \verb`c/instruction.h`


Instruction is also known as {Machine Instruction}.

\subimport{ASM/Instruction}{Instruction-i8086}

\section{Difference}

\begin{itemize}
\item Netwide ASM (version 2.07)
\item Microsoft Macro ASM (6.15)
\item GNU-AS
\end{itemize}

AASM / NASM specifies that all labels are addresses but accessing, which is different from that of C, whose array labels indicate objects but addresses.
For some assemblers, there are for a same value:

\verb|[A+B+1]|; \verb|[A+OFFSET(B)+1]|; \verb|1[A][B]|.

And there are for another value:

\verb|A|; \verb|[A]|; \verb|[OFFSET A]|.

\refer{Magice} uses AASM, which is a middle language, supporting automatically inline cross-platform AASM. If native assembly statements are used, the program will be degraded and deprived cross-platform property.

\section{Bootstrap}

\subsection{Bootstrap and Reboot Method}

%{TODO}

\subsection{Bootstrap Demonstration}\labex{Bootstrap Demonstration} \

\verb`test_floppy` test in the floppy;

\verb`test_hdisk` test in the hard-disk;

\verb`test_cdisc` is \It{to be done} for compact optical disc;

$Disk = Hard-disk + Floppy + Compact-Disc$.

\paragraph{bootfi} \

Boot FAT-Compatible Wait-instant-input. Bootstrap from a disk and you can input the parameters instantly, from \It{Arinae Codes}.

\B{Media}: \verb|Dynamic| Floppy; Hdisk;

\paragraph{boothk}\ % hdisk key-wait

Boot Wait-key. From early \It{Mecocoa}.

\B{Media}: Hdisk;

\paragraph{bootfka}\

Boot FAT-Compatible [Wait-key/File-Autorun]. From \It{Mecocoa} since 2024-Jan-29.

\B{Media}: \verb`Macro` Floppy; Hdisk;

\paragraph{bootfx}\

Boot FAT-Compatible Flat-Prot-32-Mode. Enter Simple Flat-32 Environment.

\B{Media}: Floppy;

\paragraph{FlatMbr}\


\B{Media}: Hdisk;

\begin{itemize}
	\item Paging and Segment in flat style;
\end{itemize}


\section{Platforms}

\subsection{Intel x86 and AMD64}

\subsubsection{8086-BIOS Interrupt Macro Package : Kasha}

Package Name: \verb|Kasha| % for her cherishing <Yuuki Yuuna ha Yuusha de Aru>

Mode: Real16


Symbolic Style
\begin{itemize}
	\item Fix style: \verb`BioStrPrint = BIOS String Print`
	\item Sub style: \verb`BIOS.Print`
\end{itemize}

Macro Calling Style
\begin{itemize}
	\item Direct BIN e.g. \verb`_PRINTF SYMBOL`
	\item Linked OBJ e.g. \verb`DIVX MACRO`
\end{itemize}


\subparagraph{n\_arith} \

\subsubparagraph{External Basic Operation} \

\B{DIVX} \
\\
\B{Preset}: DS:EBX=dest STACK \\
\B{Parameter}: BL or BX \\
\B{Return}: EBX+=sectors*512 \\
\B{Description}: \\
BL: AX/BL=AX...DL \\
BX: DX`AX/BX=CX`AX...DX \\
\B{Attribute}: CPU (i8086) \\
\B{Nestcall}: \It{None}

\subsubparagraph{Char Code} \

\B{BCD2ASCII} \
\\
\B{Preset}: AL \\
\B{Parameter}: \It{None} \\
\B{Return}: $AH`AL$ \\
\B{Description}: \\
\B{Attribute}: CPU (i8086) \\
\B{Nestcall}: \It{None}

\subparagraph{n\_cpuins} \

\B{CpuBrand} \
\\
\B{Preset}: DS:ESI=DATA48 \\
\B{Parameter}: \It{None} \\
\B{Return}: $VOLA(EAX,EBX,ECX,EDX,ESI,EDI)$ \\
\B{Description}: Get CPU Brand Information \\
\B{Attribute}: CPU (586+) \\
\B{Nestcall}: \It{None}



\subparagraph{n\_osdev} \

\B{Addr20Enable} \
\\
\B{Preset}: \\
\B{Parameter}: \\
\B{Return}: $AL=Port[0x92]$ \\
\B{Description}: Enable A20 to avoid automatic setting zero.\\
\B{Attribute}: CPU (i8086) \\
\B{Nestcall}: \It{None}

\B{GDTDptrStruct} \
\\
\B{Preset}: \\
\B{Parameter}: EAX BASEADDR, EBX LIMIT, ECX PROPERTY\\
\B{Return}: $EDX`EAX=FULL_DESCRIPTOR Vola{C,B}$ \\
\B{Description}:  \\
\B{Attribute}: CPU (386+) \\
\B{Nestcall}: \It{None}

\B{GateStruct} \
\\
\B{Preset}: \\
\B{Parameter}: EAX=OffsetOfSegment, BX=SegDptr, CX=Property\\
\B{Return}: $Vola(ecx,ebx) (edx`eax=Dptr)$ \\
\B{Description}:  \\
\B{Attribute}: CPU (386+) \\
\B{Nestcall}: \It{None}

\subparagraph{n\_pseudo} \

\B{File} \
\\
\B{Description}:  Use at the beginning of the file.\\

\B{Endf} \
\\
\B{Description}:  Use at the end of the file.\\

\B{DefineStack16} \
\\
\B{Preset}: \\
\B{Parameter}: SS, SP\\
\B{Return}: $Vola{AX}$ \\
\B{Description}:  \\
\B{Attribute}: BIT (16) \\
\B{Nestcall}: \It{None}

\subparagraph{n\_timer} \

\subsubparagraph{Date} \

\B{GetMoexDayIdentity} \
\\
\B{Preset}: ax year, cx month STACK DS->DataSegm\\
\B{Parameter}: ?SI String, AH/I Style\\
\B{Return}: $BX=PastDays DL=HowManyDaysInTheMonth$ \\
\B{Description}: Get the days between month 1st 00:00 through herday 24:00.\\
\B{Attribute}: BIT (16) \\
\B{Nestcall}: \It{None}

\subsubparagraph{Time} \

\B{TimerInit16} \
\\
\B{Preset}: STACK\\
\B{Parameter}: CS/I CodeSeg, RoutintINT70\\
\B{Return}: \\
\B{Description}: \\
\B{Attribute}: BIT (16) \\
\B{Nestcall}: \It{None}

\B{TimerReadTime} \
\\
\B{Preset}: after calling TimerInit16\\
\B{Parameter}: \\
\B{Return}: $HH:MM:SS \rightarrow DH:DL:AL$\\
\B{Description}: \\
\B{Attribute}: BIT (16) \\
\B{Nestcall}: \It{None}

\subparagraph{n\_video} \

Now only for 8025 mode.

\subsubparagraph{Cursor Operation} \

\B{ConCursor} \
\\
\B{Preset}: \\
\B{Parameter}: $AX posi=\sim$ \\
\B{Return}: \\
$\%1==~ Get: AX=AfterPosition Volatile\{DX\}$\\
$\%1!=~ Set: Volatile\{AX, DX\} BX=\%1=AfterPosition$\\
\B{Description}: Get or set the cursor.\\
\B{Attribute}: BIT (16) CPU (i8086) \\
\B{Nestcall}: \It{None}


\subsubparagraph{Screen Other Operation} \

\B{ConRoll} \
\\
\B{Preset}: ES=VideoBufferSegment(0xB800)\\
\B{Parameter}: $AL rows=1 yo \left[1~video_rows\right]$ \\
\B{Return}: $Volatile\{AX, CX, SI, DI\}$\\
\B{Description}: Roll the screen.\\
\B{Attribute}: BIT (16) CPU (i8086) \\
\B{Nestcall}: \It{None}


\subsubparagraph{Print Collection} \

\B{ConPrintChar} \
\\
\B{Preset}: ES=VideoBufferSegment(0xB800) STACK\\
\B{Parameter}: $AL/I char > \sim AH/I style=0x07 > \sim BX/I posi=\sim$ \\
\B{Return}: $BX=NextPosi(automatic mode)$\\
\B{Description}: Print a char on screen. \\
If bx is invalid or out of the range, bx will be set 0. \\
``posi'' starts from 0. \\
Special chars for automatic position mode: \verb'\n', \verb'\r'\\
\B{Attribute}: BIT (16) CPU (i8086) \\
\B{Nestcall}: ConCursor (2 versions)

\B{ConPrint} \
\\
\B{Preset}: ES=VideoBufferSegment(0xB800) DS=SegofString STACK\\
\B{Parameter}: $?SI String, AH/I\ Style$ \\
\B{Return}: $BX=NextPosi\ Vola{?AX}$\\
\B{Description}: Print a char on screen. \\
Accept \verb'\n', \verb'\r' for controlling and \verb'\0' for end.\\
\B{Attribute}: BIT (16) CPU (SI:i8086, ESI:386+) \\
\B{Nestcall}: ConCursor (2 versions)

{RFE221530} \It{ArinaMgk replace PUSHAD and POPAD to meet 8086 mode};

\B{ConPrintWord} \
\\
\B{Preset}: STACK\\
\B{Parameter}: BX/I word , Proc(ConPrintChar AL,AH,~) \\
\B{Return}: \\
\B{Description}: \\
\B{Attribute}: BIT (16) CPU (i8086) \\
\B{Nestcall}: ConCursor yo ConPrintChar(Extern)

\B{ConPrintDword} \
\\
\B{Preset}: STACK\\
\B{Parameter}: EBX/I word , Proc(ConPrintChar AL,AH,~) \\
\B{Return}: \\
\B{Description}: \\
\B{Attribute}: BIT (16) CPU (386+) \\
\B{Nestcall}: ConCursor yo ConPrintChar(Extern)

