
The definition and declaration of function, to see \refer{Function}.

Definition reserves room for variables, but declarations do not.
Usually a definition includes the respective declaration.

Someone call both “declaration”: The former is “Definition Declaration” and the later is “Reference Declaration”. From this perspective, declaration can be a kind of definition, and vice versa. % yo 20240308

When it comes to linkage, there is usually One Definition Rule (ODR) \textemdash only an object can own the entity of the specific object. %单定义规则
Inline function are beyond the rule, for the header can implement them without any worries about duplication.

Magice(Default configuration) and {Rust} will take the last definition for the near next code in the scope.

\It{The declaration is a kind of pseudo-instruction.} \textemdash{Arina, RFA23}.

\paragraph{non-function objects for $C,C++,M$}

\B{Experiment - Slave statements cannot be declarations}%REZ07
\lstset{style=GlobalC}
\begin{lstlisting}[language=C]
	if (...) long long i;
\end{lstlisting}

\B{Experiment - C/C++ - Tentative Linkage 1}
Below are codes in a global (out of any grace-scope) scope, where assumes N is an immediate integer. Results are done by \It{dosconio} with specific version.
\lstset{style=GlobalC}
\begin{lstlisting}[language=C]
	int i1 = N;        // definition, external linkage
	static int i2 = N; // definition, internal linkage
	extern int i3 = N; // definition, external linkage
	int i4;            // tentative , external linkage
	static int i5;     // tentative definition, internal linkage
	int i1;            // refers to line-1 (valid tentative definition)
	int i2;            // BAD (error in GCC, but pass for MSVC)
	int i3;            // refers to line-3
	int i4;            // refers to line-4
	int i5;            // BAD (like line-7)
	extern int i1;     // refers to previous external
	extern int i2;     // refers to previous internal
	extern int i3;     // refers to previous external
	extern int i4;     // refers to previous external
	extern int i5;     // refers to previous internal
	// line-7 error: non-static declaration of 'i2' follows static declaration
	// line-8 warning: 'i3' initialized and declared 'extern'
\end{lstlisting}

For {C}, the code is bad, but OK for {CPP}. For \It{Magice}, this will consider an operation with effect for stack in reality.

\B{EXAMPLE - C/C++ - Tentative Linkage 2}
If at the end of the translation unit containing, for example, \verb`int i[];`, the array \verb`i` still has incomplete type, the implicit initializer causes it to have 1 element, which is set to 0 on program startup.

\B{EXAMPLE - C/C++ - Tentative Linkage 3}
According to ISO/IEC C99
\lstset{style=GlobalC}
\begin{lstlisting}[language=C]
// OK in file scope
extern char a[];
char a[1];
\end{lstlisting}
\begin{lstlisting}[language=C]
// BAD in file scope
extern char* b;
char b[1];
\end{lstlisting}
Notice that if one is not in file(global) scope, the object will be in stack, so it is impossible to link for C/C++.


\paragraph{non-function objects for $Rust$} \

\B{EXAMPLE}
\lstset{style=GlobalRust}
\begin{lstlisting}[language=Rust]
let mut a: i32 = 10;
\end{lstlisting}
where \verb`mut`, \verb`:i32`, \verb`=10` 
are optional for what you want to do.


\paragraph{$struct/union$ for $C,C++$}

We can define them later than their declaration. For example,
\lstset{style=GlobalC}
\begin{lstlisting}[language=C]
	struct struc_a_t;
	typedef struc_a_t struc_b_t;
	// ... before using the body of `struc_a_t'
	struct struc_a_t { /*...*/ }
\end{lstlisting}


