
Below are relation between international standards and UNISYM.

The recommended standard sets are:
\begin{itemize}
\item CPL: ISO-C99
\item CPP: ISO-C++11/C++17
\end{itemize}

%{TODO} add ref-link for each:
\begin{center}\begin{longtable}{|c|c|}
\hline
\textbf{ISO/IEC C99\cite{StdC99}} & \textbf{header} \\
\hline\endfirsthead\hline
\textbf{ISO/IEC C99\cite{StdC99}} & \textbf{header} \\
\hline\endhead\hline\endfoot\hline\endlastfoot
assert.h & uassert.h \\\hline
ctype.h & \T{c/uctype.h} \\\hline
stdbool.h(C99+) & \T{c/ustdbool.h} \\\hline
string.h & ustring.h \\\hline
limits.h & \multirow{3}{*}{archit.h by stdinc} \\\cline{1-1}
stddef.h & \\\cline{1-1}
math.h & \\\hline
stdint.h & \multirow{2}{*}{integer.h by stdinc} \\\cline{1-1}
inttypes.h & \\\hline
float.h & floating.h by stdinc \\\hline
complex.h(C99+) & \multirow{3}{*}{number.h} \\\cline{1-1}
fenv.h(C99+) & \\\cline{1-1}
tgmath.h(C99+) & \\\hline
errno.h & \multirow{2}{*}{msgface.h} \\\cline{1-1}
signal.h &  \\\hline
locale.h & loc.h \\\hline
setjmp.h & \multirow{2}{*}{supple.h by stdinc} \\\cline{1-1}
stdarg.h & \\\hline
time.h & datime.h \\\hline
stdio.h & consio.h \\\hline
wchar.h(C95+) & \multirow{2}{*}{widechar.h} \\\cline{1-1}
wctype.h(C95+) & \\\hline
\end{longtable}\end{center}

\subsection{Supply \mbox{-} \B{Qt}}

If user would like to use a object-style program although it may import some red tapes, try the example:

1. Add one property in the fields.
\begin{lstlisting}[language=C++]
class WidgetMain : public QWidget
{
public:
	uni::String title;
\end{lstlisting}

2. Bind the action method.
\begin{lstlisting}[language=C++]
void WidgetMain::TitleSet(uni::String* str) {
	this->setWindowTitle(str->reflect());
}

WidgetMain::WidgetMain(QWidget *parent) : QWidget{parent} {
	unibind(title, WidgetMain::TitleSet, unipara(1));
	title = "Title";
\end{lstlisting}

Then we can use \verb|win.title = "Nihao!";| instead of \verb|win->setWindowTitle("Nihao!");|.

\subsection{Debug}
\B{Header}: auto-inc-by stdinc.

\subsection{Static Flag}
\B{Header}: auto-inc-by stdinc.

\subsection{Reference}
\B{Header}: \verb`cpp/reference`

A memory unit for complex memory or memory-reflected devices, which allows bit-wise operations.
The size of {the unit} and {the address of the unit} depends on the PCU chip.
The index of binary digit is from the LSB and using little endian.

\B{bitof}: give a binary digit in the form of \T{bool}. 

\B{setof}: set a binary digit high or low level. If the second level parameter omitted, it set the digit (high).

\B{rstof}: reset a binary digit to low level.

\paragraph{Pointer}

\B{Header}: \verb`cpp/pointer`

help avoid using "*" but \verb`Point<>` template, and also enjoy auto releasing the resources.

\subsection{Supple}

\B{Header}: (auto-inc-by \verb`stdinc`) \verb`c/system/supple.h`

\B{EXAMPLE - Greedy Parameter Using}

\begin{lstlisting}[language=C]
void print_numbers(int count, ...) {
	Letpara(args, count);
	for0 (i, count) {
		printf("%x\n", para_next(args, int));
	}
}
\end{lstlisting}

\B{EXAMPLE - Self-Defined Control Pointer}

\begin{lstlisting}[language=C]
#include <stdinc.h>
static Retpoint env;
void test_func() {
	JumpPoint(&env, "ciallo");
}
int main() {
	pureptr_t val;
	val = MarkPoint(&env);
	if (!val) {
		test_func();
	} else {
		printf("Returned from longjmp, val = %s\n", val);
	}
}
/*
aasm -f win32 %ulibpath%\asm\x86\jump.asm -o setjmpS.o -g
gcc -m32 -o a.exe test.c setjmpS.o -g -I%uincpath%
EXPECTED>a.exe
Returned from longjmp, val = ciallo
*/
\end{lstlisting}
