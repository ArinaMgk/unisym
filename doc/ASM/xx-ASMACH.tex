\labex{ASM}

\section{Directive}

\subsection{Base Compatible Platform}

Usually, this is set by the category of the respective processor.
There are examples for Intel$^{\textregistered}$80386.

\B{for NASM}

\T{CPU 386} or \T{[CPU 386]}

\B{for MASM}

\T{.386}


\subsection{Calling and Memory Method}

\B{for MASM}

\verb`.model flat, stdcall` 

flat: a 4GB segment is shared for code and data.

\section{Instruction}
\B{Header}: \verb`c/instruction.h`


Instruction is also known as {Machine Instruction}.

\subimport{ASM/Instruction}{Instruction-i8086}

\section{Difference}

\begin{itemize}
\item Netwide ASM (version 2.07)
\item Microsoft Macro ASM (6.15)
\item GNU-AS
\end{itemize}

AASM / NASM specifies that all labels are addresses but accessing, which is different from that of C, whose array labels indicate objects but addresses.
For some assemblers, there are for a same value:

\verb|[A+B+1]|; \verb|[A+OFFSET(B)+1]|; \verb|1[A][B]|.

And there are for another value:

\verb|A|; \verb|[A]|; \verb|[OFFSET A]|.

\refer{Magice} uses AASM, which is a middle language, supporting automatically inline cross-platform AASM. If native assembly statements are used, the program will be degraded and deprived cross-platform property.




