

\textbf{UNISYM} (Universal Symbolic Library, aka `Uniformed Symbolic Library') originates from the abstraction of a lossless computing project (Silver-Garden-Arithmetic), which is transformed into \textbf{COTLAB}.

\textbf{Features of UNISYM}

Portability, Coupling and Practicality. Portability is for showing the uniform; Coupling is for expressing the systematic logic; Practicality is for making sense.

\begin{itemize}
	\item Architectures (Dinah, x86, RISCV64, ARM, ...)
	\item Compilers (MC-AASM, GCC, MSVC, ...)
	\item Operating Systems (Mecocoa, Windows, Linux, ...)
	\item Programming Languages (Magice, \verb|C|, \verb|C#|, ...)
\end{itemize}

\section{Locale}

\begin{center}\begin{table}[h!]
	\begin{tabular}{|l|l|l|}
	\hline
	\rowcolor[rgb]{0.95, 0.975, 1}\textbf{REGION} & \textbf{PHONE-HEADER} & \textbf{ANSI} \\ \hline
	China Mainland & +86 & GBK(ZH-CN) \\ \hline
	%{TODO}
	\end{tabular}
\end{table}\end{center}

\section{Series \mbox{-} {MCUDEV}}

\section{Library \mbox{-} Electronics}

\section{Library \mbox{-} \B{C\#} Library}

\section{Library \mbox{-} \B{Rust} Crate}

\section{Library \mbox{-} Assembly Macros}

The section does not include \It{Kasha}. For AASM, Microsoft Macro Assembler, GNU-AS, Netwide Assembler.

\section{Supply \mbox{-} \B{Qt}}

\section{Charset}

You may need the header \verb|c/multichar.h| or \verb|c/widechar.h|.

\subsection{ASCII}

\subsection{Unicode}



\textbf{CscUTF} Convert UTFx to another UTF format.

\verb|stduint CscUTF(byte from, byte to, const pureptr_t src, stduint slen, pureptr_t* des);|

If something fail, this will return `NONE', or return the counts of effective bytes of the string in the Unicode format.




% Last sections

\section{Wel}
\subimport{.}{Wel}

\section{Witch Graphic}
\subimport{.}{Witch}

\section{Wizard}
\subimport{.}{Wizard}

\section{Demonstration}

\begin{itemize}
	\item `Bootstrap' to see \refer{Bootstrap Demonstration}.
\end{itemize}

\section{Utility}

\section{Option Switch}

% _MCU_

% _SUPPORT_ {device/bits}

% _INC_

% _CALL_


% _ARC_ and _OPT_

% _MCCA

\section{Data and Table}

% _tab_ prefix
