

\textbf{UNISYM} (Universal Symbolic Library, aka `Uniformed Symbolic Library') originates from the abstraction of a lossless computing project (Silver-Garden-Arithmetic), which is transformed into \textbf{COTLAB}.

\textbf{Features of UNISYM}

Portability, Coupling and Practicality. Portability is for showing the uniform; Coupling is for expressing the systematic logic; Practicality is for making sense.

\begin{itemize}
	\item Architectures (Dinah, x86, RISCV64, ARM, ...)
	\item Compilers (MC-AASM, GCC, MSVC, ...)
	\item Operating Systems (Mecocoa, Windows, Linux, ...)
	\item Programming Languages (Magice, \verb|C|, \verb|C#|, ...)
\end{itemize}


\section{Charset}

You may need the header \verb|c/multichar.h| or \verb|c/widechar.h|.

\subsection{Unicode}



\textbf{CscUTF} Convert UTFx to another UTF format.

\verb|stduint CscUTF(byte from, byte to, const pureptr_t src, stduint slen, |\\
\verb|pureptr_t* des);|

If something fail, this will return `NONE', or return the counts of effective bytes of the string in the Unicode format.


