
We usually consider procedure or sub-routine only exist in Assembly. % 过程
Someone call the function returning nothing procedure. %函数

\subsubsection{Calling Statement}

Assume \verb`f` is a pointer to some function, or from an entity \verb`void f() {...}`, then \verb`f(...)` or \verb`(*f)(...)` can call the function, where \verb`...` stands for arguments actually. Even none-* or multi-* can call it, like \verb`(*****f)()`. % expi by dscn, 20240905

\subsubsection{Parameter and Argument}

\paragraph{For C or C++}\

Because pointers are in a size, it is OK to write \verb`f(int arr[1][2]) {...}` as \verb`f(int arr[][2]) { ... }` or \verb`f(int (*arr)[2]) { ... }`.
For multi arguments, the evaluation order (before calling) is decided by the compiler, and usually the passing order is from right to left (under CDECL, CPL Declaration Rule).

When calling the function, the item filled in the place of the respective parameter is called `argument'. Hence parameter is {formal argument}.

The form of parameter passing: reference, stack(of caller or callee), register.

The \B{storage-class specifier} for a parameter declaration is only \verb`register`.

\subsubsection{Calling Convention}\labex{Calling}

\subsubsection{Register BP for Entering and Leaving Function}

%{} for functions that have arguments or local variables, ...

\subsubsection{Allocation of Stack}

This is for the parameters and locale variables. Whether the arguments are aligned with bytes depends on the compiler. For Visual C++ in Win32, it usually with multi-bytes (a \verb`dword`), while it may be  only one for GCC. Try like this if we want to assure:

\lstset{style=GlobalC}
\begin{lstlisting}[language=C]
	// ASCII C TAB4 CRLF
	// Docutitle: (Demo) 
	// Codifiers: @dosconio
	// Attribute: Endian(Little)(for %x occuper)
	#include <stdio.h>
	int main() {
		char a, b;
		printf("%x %x", &a, &b);
	}
\end{lstlisting}

\subsubsection{Inline Function for $C/C++$}

This appeared when the keyword \refer{keyword.inline} appeared.
In a hosted environment, inline cannot specify function \It{main}.

\B{EXAMPLE - Avoid side-effect}
\lstset{style=GlobalC}
\begin{lstlisting}[language=C]
inline MIN(int a, int b){if (a < b)return a; else return b;}
#define min(x,y)((x)<(y)?(x):(y))
\end{lstlisting}
Assume \verb|a| assigned "1". Then
\verb`min(++a,b);` makes \verb|a| equals 3, while
\verb`MIN(++a,b);` makes \verb|a| equals 2.

