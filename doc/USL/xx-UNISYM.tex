

\textbf{UNISYM} (Universal Symbolic Library, aka `Uniformed Symbolic Library') originates from the abstraction of a lossless computing project (Silver-Garden-Arithmetic), which is transformed into \textbf{COTLAB}.

The `Library' below are specific modules, for common modules described in \refer{CLOVER}.

\textbf{Features of UNISYM}

Portability, Coupling and Practicality. Portability is for showing the uniform; Coupling is for expressing the systematic logic; Practicality is for making sense.

Loosely coupled design are applied to our Outputs, but Tightly for the library herself.

\begin{itemize}
\item Architectures (Dinah, x86, RISCV64, ARM, ...)
\item Compilers (MC-AASM, GCC, MSVC, ...)
\item Operating Systems (Mecocoa, Windows, Linux, ...)
\item Programming Languages (Magice, \verb|C|, \verb|C#|, ...)
\end{itemize}

\subsection{NOTICE when Modifying}

\verb`_CALL_C` for Calling, \verb`extern "C"` for Linkage


\B{Header Included}:
If you develop something on host-environment, include `stdinc.h` for C/C++;
if you develop MCU-program, use the special file like `STM32F1` for C++ (See more in \refer{QANRUS})...

\section{Notice}

\B{Root Header for User}
It is recommended that users use only header files whose directory is the root including path. For example, we should include \verb`<stdinc.h>` but \verb|system/alice.h|.


\section{Locale}
\B{Header}: \verb`c/loc.h`

\begin{center}\begin{table}[h!]
	\begin{tabular}{|l|l|l|}
	\hline
	\rowcolor[rgb]{0.95, 0.975, 1}\textbf{REGION} & \textbf{PHONE-HEADER} & \textbf{ANSI} \\ \hline
	China Mainland & +86 & GBK(ZH-CN) \\ \hline
	%{TODO}
	\end{tabular}
\end{table}\end{center}

\section{Dependence Map}

\section{Library \mbox{-} Electronics}

\section{Library \mbox{-} \B{C\#} Library}

\section{Library \mbox{-} \B{Rust} Crate}

\section{Library \mbox{-} Assembly Macro}

The section does not include \It{Kasha}. For AASM, Microsoft Macro Assembler, GNU-AS, Netwide Assembler.

%reference www.ic.unicamp.br/~ducatte/mc404/Lampiao/docs/dosints.pdf

\subsection{IBM Compatible Interface}

\subsection{MS-DOS Interface}

\section{Supplement for C, C++}
\subimport{.}{Supple}

\section{Charset}

You may need the header \verb|c/multichar.h| or \verb|c/widechar.h|.

\subsection{ASCII}

\subimport{Charset}{ASCII}

\subsection{Unicode}



\textbf{CscUTF} Convert UTFx to another UTF format.

\verb|stduint CscUTF(byte from, byte to, const pureptr_t src, stduint slen, |\\
\verb|pureptr_t* des);|

If something fail, this will return `NONE', or return the counts of effective bytes of the string in the Unicode format.




\section{Console}

Console Input and Output: $CONIO$ > $CONSIO$ > $CONSOLEIO$.


\section{File System and Format}

\subsection{File System}

\subsubsection{FAT12}

\B{Header}: \verb`c/format/FAT12.h`

% None

\subsection{File Format}

\subsubsection{ELF}

\B{Header}: \verb`c/format/ELF.h`

% Struct ...

% Func ...



\section{Signal and Logging}

\B{Header}: \verb`c/msgface.h`




% Last sections

\section{Wel}
\subimport{.}{Wel}

\section{Witch Graphic} \labex{Witch}
\subimport{.}{Witch}

\section{Wizard}
\subimport{.}{Wizard}

\section{Demonstration}

\begin{itemize}
	\item `Bootstrap' to see \refer{Bootstrap Demonstration}.
	\item `Hello' (Print a string to standard output) by each programming language. 
	% According to the record, Brian Kernighan wrote Hello-world for B Programming Language, and later this became widely known in "The C Programming Language" written by Professor Kernighan & Ritchie in 1978.
\end{itemize}

\section{Utility}

\begin{itemize}
\item \B{args}
\item \B{cal}    calendar (Linux Command)
\item \B{clear}  (Linux Command)
\item \B{cpuid}
\item \B{fdump}  filedump
\item \B{ffset}  VirtualDiskCopier
\item \B{ret}    return \\
	Return the integer and echo string with new-line, usually used with \verb`&&` or \verb `||` instead of `echo'.
\item \B{segsel} SegmentSelector
\end{itemize}

\section{Output Target and Option Switch}

\subsubsection{Freestanding Environment}

Relative command options: %{TODO}

\subsubsection{Hosted Environment}

In Freestanding Environment, user can only use inline assembly, memory and arithmetic effect. However, the hosted includes at least interfaces provided by Operating System and Processor-registered mechanism, which provide much side-effect.

Usually to see demo `args':

\lstset{style=GlobalC}
\lstinputlisting[language=C, firstline=24]{../demo/utilities/args.c}

But the ISO/IEC recommend the below forms for `main' function with external symbol \verb`_main`:

\verb`int main(int argc, char *argv[]) {/* ... */}` or 
\verb`int main(void) { }`

\B{EXAMPLE - Empty-return in Startup}
\begin{lstlisting}[language=C]
// ARINA RFW4 TEST on GCC Lin32
void main() { int a=2; }
// arina@aUbt:~$ echo $?
//:1
\end{lstlisting}

Learn the appliance with the utility `args'.


\section{Builtin-Data}

\subsection{.picture and other Demonstration}

\subsection{Linked Data}

in header \verb`c/data.h`
\begin{itemize}
	\item \verb `_BITFONT_ASCII_8x5`
	\item \verb `_BITFONT_ASCII_16x8`
\end{itemize}

in header \verb`c/uctype.h`
\begin{itemize}
	\item \verb `_tab_HEXA` decimal to upper ASCII chars.
	\item \verb `_tab_hexa` decimal to lower ASCII chars.
	\item \verb `_tab_alnum_digit` ASCII chars to hexadecimal.
\end{itemize}

in header \verb`c/driver/keyboard.h`
\begin{itemize}
	\item \verb `_tab_keycode2ascii` Keyboard Scan-code to ASCII and description.
\end{itemize}

\subsection{Assembly Instructions}

%{} from NASMANL

