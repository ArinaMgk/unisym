

\textbf{UNISYM} (Universal Symbolic Library, aka `Uniformed Symbolic Library') originates from the abstraction of a lossless computing project (Silver-Garden-Arithmetic), which is transformed into \textbf{COTLAB}.

\textbf{Features of UNISYM}

Portability, Coupling and Practicality. Portability is for showing the uniform; Coupling is for expressing the systematic logic; Practicality is for making sense.

\begin{itemize}
	\item Architectures (Dinah, x86, RISCV64, ARM, ...)
	\item Compilers (MC-AASM, GCC, MSVC, ...)
	\item Operating Systems (Mecocoa, Windows, Linux, ...)
	\item Programming Languages (Magice, \verb|C|, \verb|C#|, ...)
\end{itemize}

\subsection{NOTICE when Modifying}

\verb`_CALL_C` for Calling, \verb`extern "C"` for Linkage


\B{Header Included}:
If you develop something on host-environment, include `stdinc.h` for C/C++;
if you develop MCU-program, use the special file like `STM32F1` for C++...


\section{Locale}

\begin{center}\begin{table}[h!]
	\begin{tabular}{|l|l|l|}
	\hline
	\rowcolor[rgb]{0.95, 0.975, 1}\textbf{REGION} & \textbf{PHONE-HEADER} & \textbf{ANSI} \\ \hline
	China Mainland & +86 & GBK(ZH-CN) \\ \hline
	%{TODO}
	\end{tabular}
\end{table}\end{center}

\section{Series \mbox{-} {MCUDEV}}

\section{Library \mbox{-} Electronics}

\section{Library \mbox{-} \B{C\#} Library}

\section{Library \mbox{-} \B{Rust} Crate}

\section{Library \mbox{-} Assembly Macros}

The section does not include \It{Kasha}. For AASM, Microsoft Macro Assembler, GNU-AS, Netwide Assembler.

\section{Supply \mbox{-} \B{Qt}}

\section{Supply for C and C++}

Below are relation between international standards and UNISYM.

%{TODO} add ref-link for each:
\begin{center}\begin{longtable}{|c|c|}
		\hline
		\textbf{ISO/IEC C99\cite{StdC99}} & \textbf{header} \\
		\hline\endfirsthead\hline
		\textbf{ISO/IEC C99\cite{StdC99}} & \textbf{header} \\
		\hline\endhead\hline\endfoot\hline\endlastfoot
		assert.h & uassert.h \\\hline
		ctype.h & uctype.h \\\hline
		stdbool.h(C99+) & ustdbool.h \\\hline
		string.h & ustring.h \\\hline
		limits.h & \multirow{3}{*}{archit.h by stdinc} \\\cline{1-1}
		stddef.h & \\\cline{1-1}
		math.h & \\\hline
		stdint.h & \multirow{2}{*}{integer.h by stdinc} \\\cline{1-1}
		inttypes.h & \\\hline
		float.h & floating.h by stdinc \\\hline
		complex.h(C99+) & \multirow{3}{*}{number.h} \\\cline{1-1}
		fenv.h(C99+) & \\\cline{1-1}
		tgmath.h(C99+) & \\\hline
		errno.h & msgface.h \\\hline
		locale.h & loc.h \\\hline
		setjmp.h & \multirow{2}{*}{supple.h} \\\cline{1-1}
		stdarg.h & \\\hline
		time.h & datime.h \\\hline
		stdio.h & consio.h \\\hline
		wchar.h(C95+) & \multirow{2}{*}{widechar.h} \\\cline{1-1}
		wctype.h(C95+) & \\\hline
\end{longtable}\end{center}

\section{Charset}

You may need the header \verb|c/multichar.h| or \verb|c/widechar.h|.

\subsection{ASCII}

\subsection{Unicode}



\textbf{CscUTF} Convert UTFx to another UTF format.

\verb|stduint CscUTF(byte from, byte to, const pureptr_t src, stduint slen, |\\
\verb|pureptr_t* des);|

If something fail, this will return `NONE', or return the counts of effective bytes of the string in the Unicode format.




% Last sections

\section{Wel}
\subimport{.}{Wel}

\section{Witch Graphic}
\subimport{.}{Witch}

\section{Wizard}
\subimport{.}{Wizard}

\section{Demonstration}

\begin{itemize}
	\item `Bootstrap' to see \refer{Bootstrap Demonstration}.
\end{itemize}

\section{Utility}

\section{Option Switch}

\B{\_MCU\_}
\begin{itemize}
	\item \B{\_MCU\_Intel8051}\begin{itemize}
		\item \verb`_SYS_FREQ` Frequency of the system.
	\end{itemize}
	%{TODO}
\end{itemize}

\B{\_SUPPORT\_} %{device/bits}
\begin{itemize}
	\item \B{GPIO}: Usually for MCU.
	\item \B{Port8}: Usually for Personal Computers.
\end{itemize}

\B{\_INC\_}
\begin{itemize}
	\item%{TODO}
\end{itemize}

\B{\_CALL\_} (is not Linkage)
\begin{itemize}
	\item%{TODO}
\end{itemize}

\B{\_ARC\_} and \B{\_OPT\_}
\begin{itemize}
	\item%{TODO}
\end{itemize}

\B{\_MCCA\_} and Operating System
\begin{itemize}
	\item \verb|0x8616| x86, bits 16
	\item \verb|0x8632| x86, bits 32
	%\item \_Win32
	%\item \_Win64
	%\item \_WinNT
\end{itemize}

\section{Data and Table}

% _tab_ prefix
