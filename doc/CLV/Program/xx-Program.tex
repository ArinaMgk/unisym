
\subsection{Bridge between Machine-code and Advanced-language}

The concept of `state machine' is widely accepted, for it describing the effect of the instructions, which can help us to get numeric register-effects and other side-effects.

For example, the rules of conversion from function calling, entering(including allocating stack and others), leaving and evaluation to assembly implementation should be specific.

More to see
\begin{itemize}
	\item \refer{Calling}
\end{itemize}

\subsection{Syntax, Semantics and Statement} % Grammar
\subimport{CLV/Program}{Statement}

\subsection{Scope and Linkage}

\subsection{Identifier and Token Parsing}
\subimport{CLV/Program}{Identifier}

\subsection{Keyword and Operator Parsing}
\subimport{CLV/Program}{Keyword}

for Std-C99, Single(Suffix then Prefix) $>$ \textbf{Double} $>$ Triple $>$ Assignment $>$ Half-Statement(Comma) .


\begin{center}\begin{longtable}{|c|l|l|}
	\hline
	\textbf{Precedence/} & \textbf{Operator} & \textbf{Description} \\
	\textbf{Associativity} & \textbf{} & \textbf{} \\
	\hline
	\endfirsthead
	\hline
	\textbf{P/A} & \textbf{Operator} & \textbf{Description} \\
	\hline
	\endhead
	\hline
	\endfoot
	\hline
	\endlastfoot
	$\mathop{1}\limits ^{\rightarrow}$ 
	& \verb|++|, \verb|--| & Suffix/postfix increment and decrement \\
	\cline{2-3}
	& \verb|()| & Function call \\
	\cline{2-3}
	& \verb|[]| & Array subscripting \\
	\cline{2-3}
	& \verb|.| & Structure and union member access \\
	\cline{2-3}
	& \verb|->| & Structure and union member access through pointer \\
	\cline{2-3}
	& \verb|(type0){...}| & Compound literal (C99) \\
	\hline
	$\mathop{2}\limits ^{\leftarrow}$ 
	& \verb|++|, \verb|--| & Prefix increment and decrement \\
	\cline{2-3}
	& \verb|+|, \verb|-| & Unary plus and minus \\
	\cline{2-3}
	& \verb|!|, \verb|~| & Logical NOT and bitwise NOT \\
	\cline{2-3}
	& \verb|(type0)| & Cast conversion \\
	\cline{2-3}
	& \verb|*| & Indirection (dereference) \\
	\cline{2-3}
	& \verb|&| & Address-of l-value united in byte\\
	\cline{2-3}
	& \verb|sizeof| & Size-of united in byte\\
	\cline{2-3}
	& \verb|_Alignof| & Alignment requirement (C11) \\
	\hline
	$\mathop{3}\limits ^{\rightarrow}$ & \verb|*|, \verb|/|, \verb|%| & Multiplication, division, and remainder \\
	\hline
	$\mathop{4}\limits ^{\rightarrow}$ & \verb|+|, \verb|-| & Addition and subtraction \\
	\hline
	$\mathop{5}\limits ^{\rightarrow}$ & \verb|<<|, \verb|>>| & Bitwise left shift and right shift \\
	\hline
	$\mathop{6}\limits ^{\rightarrow}$ & \verb|<|, \verb|<=| & For relational operators < and ≤ respectively \\
	\cline{2-3}
	& \verb|>|, \verb|>=| & For relational operators > and ≥ respectively \\
	\hline
	$\mathop{7}\limits ^{\rightarrow}$ 
	& \verb|==|, \verb|!=| & For relational = and ≠ respectively \\
	\hline
	$\mathop{8}\limits ^{\rightarrow}$ 
	& \verb|&| & Bitwise AND \\
	\hline
	$\mathop{9}\limits ^{\rightarrow}$ 
	& \verb|^| & Bitwise XOR (exclusive or) \\
	\hline
	$\mathop{10}\limits ^{\rightarrow}$ 
	& \verb!|! & Bitwise OR (inclusive or) \\
	\hline
	$\mathop{11}\limits ^{\rightarrow}$ 
	& \verb|&&| & Logical AND \\
	\hline
	$\mathop{12}\limits ^{\rightarrow}$ 
	& \verb!||! & Logical OR \\
	\hline
	$\mathop{13}\limits ^{\leftarrow}$
	& \verb!?:! & Ternary conditional  \\
	\hline
	$\mathop{14}\limits ^{\leftarrow}$
	& \verb|=| & Simple Assignment (Below are Compound) \\
	\cline{2-3}
	& \verb|+=|, \verb|-=| & Assignment by sum and difference \\
	\cline{2-3}
	& \verb|*=|, \verb|/=|, \verb|%=| & Assignment by product, quotient, and remainder \\
	\cline{2-3}
	& \verb|<<=|, \verb|>>=| & Assignment by bitwise left shift and right shift \\
	\cline{2-3}
	& \verb|&=|, \verb|^=|, \verb!|=! & Assignment by bitwise AND, XOR, and OR \\
	\hline
	$\mathop{15}\limits ^{\rightarrow}$ & \verb|,| & Comma \\
	\hline
\end{longtable}\end{center}

Address-of is called by reference in some books. This does not for bit-field and register objects.

%{TODO} \verb|&| returns a l- or r- value address.
%{TODO} used with ‘*’ ‘[]’ is like ‘LEA’ instruction.




\subsection{Preprocessing and Parsing}
\subimport{CLV/Program}{Parsing}

\subsection{Procedure, Function and Method}
\labex{Function}
\subimport{CLV/Program}{Function}

\subsection{Standard, Style and Arinae Covenant}
\subimport{CLV/Program}{Standard}

\subsection{Target and Output Format}

\subsubsection{Freestanding Environment}

Relative command options: %{TODO}

\subsubsection{Hosted Environment}

In Freestanding Environment, user can only use inline assembly, memory and arithmetic effect. However, the hosted includes at least interfaces provided by Operating System and Processor-registered mechanism, which provide much side-effect.

Usually to see demo `args':

\lstset{style=GlobalC}
\lstinputlisting[language=C, firstline=24]{../demo/utilities/args.c}

But the ISO/IEC recommend the below forms for `main' function with external symbol \verb`_main`:

\verb`int main(int argc, char *argv[]) {/* ... */}` or 
\verb`int main(void) { }`

\B{EXAMPLE - Empty-return in Startup}
\begin{lstlisting}[language=C]
// ARINA RFW4 TEST on GCC Lin32
void main() { int a=2; }
// arina@aUbt:~$ echo $?
//:1
\end{lstlisting}

Learn the appliance with the utility `args'.



\subsection{Assertion and Exception}
\subsubsection{Assertion}
\B{Header}: \verb`c/uassert.h`

\B{assert}(basic): If basic is zero or null, the program will exit with abnormal feedback.

\subsubsection{Interrupt}
\B{Header}: \verb`cpp/interrupt`


\paragraph{GIC and NVIC} \

Generic and Nested Vector Interrupt Controller

\paragraph{EXTI - External Interrupt} \


\subsubsection{Exception}



\subsection{Parallel-Thread and Multi-process}

\subsubsection{Spin-lock}

\subsection{Binary Machine Operation}
\B{Header}: \verb`c/binary.h`

\paragraph{CPUID Operation} \
\B{Header}: \verb`c/cpuid.h`


\subsection{Special Notion and Adaptation}







