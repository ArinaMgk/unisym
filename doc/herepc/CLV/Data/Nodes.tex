
The chain of a node contains a series of nodes. There are usually discrete so we need not allocate them at once.

\begin{center}
\begin{longtable}{c|c|c|l}%[h!]
		\caption{Comparison between Non-annulus Chain}
		\label{tab:table_na_nodes} \\
		%\begin{tabular}{c|c|c|l} if no using longtable
		\hline \textbf{NODE} & \textbf{DNODE} & \textbf{NNODE} & \textbf{Puts}\\
		chain\_t Chain & dchain\_t Dchain T[2] & nchain\_t Nchain & {} \\
		\hline
		\endfirsthead
		\multicolumn{4}{c}%
		{{\bfseries \tablename\ \thetable{} -- continued from previous page}} \\
		\hline \textbf{NODE} & \textbf{DNODE} & \textbf{NNODE} & \textbf{Puts}  \\  \hline  
		\endhead
		\hline \multicolumn{4}{r}{{Continued on the next page...}} \\
		\endfoot
		\hline \hline
		\endlastfoot
		
		\multicolumn{2}{c|}{Linear: $ArrayTrait$, $IterateTrait$} & Nested & \textit{(Form)} \\ 
		\hline
		\multicolumn{3}{c}{\$::$ReheapString$} & \textit{(S.R.S.)} \\ % "{c}" means no- split
		\hline
		\multicolumn{3}{c}{\$::$GetExtnField()$} & {} \\
		\hline
		\multicolumn{3}{c|}{$\$Insert$} & {nod,off,(type),ex-field,(on-right)$\rightarrow$(*)} \\
		\hline
		\multicolumn{3}{c|}{$\$Remove$} & {nod,(left),fn-free$\rightarrow$(*)(*next)} \\
		\hline
		\multicolumn{3}{c|}{$\$sRelease$ $\$\$Drop$} & {since,fn-free} \\
		\hline
		\multicolumn{3}{c|}{$\$HeapFreeSimple$} & {nod} \\
		\hline
		\multicolumn{3}{c|}{\$\$$Init$} & {()} \\
		\hline
		\multicolumn{3}{c|}{\$\$::$New$ \$$New$} & {(self) or ()$\rightarrow$(*)} \\
		\hline
		\multicolumn{3}{c|}{\$\$::$Count$ \$$Count$} & {count} \\
		\hline
		\multicolumn{3}{c|}{\$\$(::)$Append$} & {(self),off,left,nod[=0 or ,type]$\rightarrow$(*new)} \\
		\hline
		\multicolumn{3}{c|}{\$\$(::)$LocateNode$} & {$\rightarrow$(*) $\|$ {index};{off,fn\_cmp=0}} \\
		\hline
		\multicolumn{3}{c|}{\$\$::$Root$ and \$\$::$RootRef$} & {$\rightarrow$(*) or (\&*)} \\
		\hline
		\multicolumn{2}{c|}{\$\$::$<<$, \$\$::$[]$} & {\$\$::$<<$, \$::$[]$} & {} \\
		\hline
		\multicolumn{2}{c|}{\$\$::$Last$} & {\$::$Youngest$} & {(*)} \\
		\hline
		\multicolumn{2}{c|}{\$\$::$Head$} & {\$::$Head$} & {(*)} \\
		\hline
		\multicolumn{2}{c|}{\$\$::$Tail$} & {\$::$Tail$} & {(*)} \\
		\hline
		\multicolumn{2}{c|}{\$\$::$Sorted$ \$\$$Sort$} & {/} & {(self),fn-cmp} \\
		\hline
		{} & \multicolumn{2}{c|}{\$::$GetTnodeField$} & {} \\
		\hline
\end{longtable}
\end{center}

a `\$' stands for a node, such as `Node', `Dnode' or `Nnode', while double dollars `\$\$' stands for the respective chain, such as `Chain', `Dchain' or `Nchain'.

\subsubsection{Node Chain : Single-Field Chain}

\paragraph{Structure} \

\lstset{style=GlobalC}
\begin{lstlisting}[language=C]
	typedef struct Node {
		struct Node* next;
		union {
			const char* addr;
			pureptr_t offs;
		};
	} Node; // measures stdint[2]
\end{lstlisting}

\lstset{style=GlobalC}
\begin{lstlisting}[language=C]
#define _MACRO_CHAIN_MEMBER \
Node* root_node;\
Node* last_node;\
struct {\
	Node* midl_node;\
} fastab;\
stduint node_count;\
stduint extn_field;\
struct {\
	bool been_sorted /* `need_sort` as para of Append */;\
} state;

// C...
typedef struct NodeChain_t {
	_MACRO_CHAIN_MEMBER
	_tofree_ft func_free;
	_tocomp_ft func_comp;
} chain_t;

// C++...
class Chain : public ArrayTrait, public IterateTrait {
protected: _MACRO_CHAIN_MEMBER /*...*/
public:
	_tofree_ft func_free;
	Chain(bool defa_free = false);
	~Chain();
	/*...*/
}

\end{lstlisting}
% Node
Trait:  \verb|ArrayTrait| \verb|IterateTrait|

\paragraph{Method} \

\textbf{\$\$::SortByInsertion}

\textbf{\$\$::getLeft}
Get the left node of the node in the chain.



.

\subsubsection{Dnode Chain : Double-Field-and-Linked Chain}

\paragraph{Structure} \

\lstset{style=GlobalC}
\begin{lstlisting}[language=C]
typedef struct Dnode {
	struct Dnode* next;
	union { char* addr; pureptr_t offs; };
	struct Dnode* left;
	union { stduint type; stduint lens; };
} Dnode; // measures stdint[4]
\end{lstlisting}

\lstset{style=GlobalC}
\begin{lstlisting}[language=C]
#define _MACRO_DCHAIN_MEMBER \
Dnode* root_node;\
Dnode* last_node;\
struct {\
	Dnode* midl_node;\
} fastab;\
stduint node_count;\
stduint extn_field;\
struct {\
	bool been_sorted /* `need_sort` as para of Append */;\
} state;
	
// C...
typedef struct DnodeChain_t {
	_MACRO_DCHAIN_MEMBER
	_tofree_ft func_free;
	_tocomp_ft func_comp;
} dchain_t;
typedef struct Tnode {
	Dnode;
	TnodeField;
} Tnode;// Magice Style
	
// C++...
class Dchain : public ArrayTrait, public IterateTrait {
protected: _MACRO_DCHAIN_MEMBER /*...*/
public:
	_tofree_ft func_free;
	Dchain(); ~Dchain();
	/*...*/
}
	
\end{lstlisting}

% Dnode
Trait:  \verb|ArrayTrait| \verb|IterateTrait|

\paragraph{Method} \

\textbf{\$\$::SortByInsertion}

\textbf{DnodeRewind}

\textbf{\$\$::Reverse}

\textbf{\$\$::Match}

\textbf{\$\$::AppendMatch}

\textbf{\$\$::Reset}

\textbf{TokenParseUnit}

\subsubsection{Mnode Chain : Mapping Chain}

\subsubsection{Nnode Chain : Nested Chain}

\paragraph{Structure} \

\lstset{style=GlobalC}
\begin{lstlisting}[language=C]
typedef struct Nnode
{
	// Dnode
	struct Nnode* next;
	union { char* addr; void* offs; };
	union { struct Nnode* left, * pare; };
	union { stduint type; stduint lens; };
	//
	struct Nnode* subf;// sub-first item
} Nnode, nnode;
\end{lstlisting}

\lstset{style=GlobalC}
\begin{lstlisting}[language=C]
// C...
typedef struct NnodeChain_t {
	Nnode* root_node;
	stduint extn_field;
	_tofree_ft func_free;
	_tocomp_ft func_comp;
} nchain_t, Nchain;

// C++...
class Nchain {
protected:
	Nnode* root_node;
public:
	_tofree_ft func_free;
	stduint extn_field;
	//
	Nchain(bool defa_free = false); ~Nchain();
};
	
\end{lstlisting}

% Nnode
Trait:  

\paragraph{Method} \


\textbf{\$::isEldest, Nnode\_isEldest}

\textbf{NnodeEldest}
Get eldest node of parallel generation of the node.

\textbf{\$\$::DivideSymbols}

NestedParseUnit, TokenOperator, TokenOperatorGroup ...

\subsubsection{Gnode Chain : Graph}

%\subparagraph{aaaa}
%\subsubparagraph {aaaaa}
%\subsubsubparagraph {bbbbb}

