
General Purpose Input Output

\B{Header}: \verb`c/port.h`, \verb`cpp/Device/GPIO`.

\B{Trait}: \verb`RuptTrait` for GPIO.

\B{Sub-Generation}: 2

Compared with other peripherals, GPIO own ``pure-virtual'' class, for the number of which is up to thousands.
So do NOT use \It{him}\verb`.RuptTrait::enInterrupt(some_hand);`.

\paragraph{Pin Method} \

\begin{itemize}
\item operator \verb`=` for reading and writing the pin

\item $getID$ $\rightarrow$ an integer, and $getParent$ \verb`()` $\rightarrow$ {Port}.

\item $setMode$ \verb`(Mode, Speed=defa_speed, autoEnClk = true)` $\rightarrow$ self

Set the working mode of GPIO.

The third parameter is left for historical reason, this may be removed in the future. In some MCU, we can config registers when the GPIO is disabled. % She remembered STM32F103VET6 

\item Set Rupt or Event Mode

\item $canMode$

Cancel working modes, destructure the device, and finally disable the working clock signal.

\item setPull, use setMode to flush pull floating

\item void Toggle()

\item void Lock()

\item bool isInput()

\item Connect %setAlternate

\paragraph{Port Method} \

operator \verb`=` for reading and writing the port (multiple pins).

\paragraph{Physical GPIOs} \

Each port contains 16 pins.

For MSP432P4, usually use GPIO 0, 1, 2, ... 8 pins ports.

\begin{longtable}{|l|c|c|c|c|c|}
	\hline \rowcolor[rgb]{0.95, 0.975, 1}
	{MCU} & 
	\B{GPIO A B C} & \B{GPIO D E} & \B{GPIO F} & \B{GPIO G H I} & \B{GPIO J}
	\\ \hline\endfirsthead
	\hline \rowcolor[rgb]{0.95, 0.975, 1}
	{(Cont.)} & 
	\B{GPIO A B C} & \B{GPIO D E} & \B{GPIO F} & \B{GPIO G H I} & \B{GPIO J}
	\\ \hline\endhead\hline\endfoot\hline\endlastfoot
	%
	STM32F1 & Ex. & Ex. & Ex. & . & .  
	\\ \hline
	STM32F4 & Ex. & Ex. & Ex. & Ex. & .  
	\\ \hline
	STM32MP13 & Ex. & Ex. & Ex. & Ex. & .  
	\\ \hline
	CW32F030 & Ex. & . & Ex. & . & .  
	\\ \hline
	MSP432P4 & Ex. & Ex. & . & . & Ex. 
	\\ \hline
\end{longtable}


\end{itemize}


