% origin K/symbol.md
%{TODO} 友情链接 [[U/constant|constant]];

\T{A a} $\alpha$\begin{itemize}
	\item {$\mathbf{A}$ Electronic Potential} % F/Electronic/Potential|矢势
	\item {$\vec{\alpha}$ } % 矢量面积
\end{itemize}

\T{B b} $\beta$\begin{itemize}
	\item ...
\end{itemize}

\T{C c} $\Psi$ $\psi$\begin{itemize}
	\item $C = \dfrac{\varepsilon·A}{d}$ %电容系数
	; where $W = \int_0^Q(\dfrac{q}{C})dq = \dfrac{1}{2} \dfrac{Q^2}{C}$ %{TEMP}
\end{itemize}

\T{D d} $\Delta$ $\delta$\begin{itemize}
	\item Electric Displacement $\vec{D}$ %电位移矢量
		where  $\vec{D}=\varepsilon_0 \vec{E} + \vec{P}$, %P 为电极化强度
		%变化的电场等效出来的电流
	\item $\delta$ % [[F/Notion/Skin-depth|趋肤深度]]
\end{itemize}

\T{E e} $\epsilon$ $\varepsilon$\begin{itemize}
	\item Electromotive Force (EMF) $E$ %电动势
\end{itemize}

\T{F f} $\Phi$ $\phi$ $\varphi$
\begin{itemize}
	\item Force
	\begin{enumerate}
		\item $\vec F = a \vec E$ %电场よの力
	\end{enumerate}
\end{itemize}

\T{G g} $\Gamma$ $\gamma$\begin{itemize}
	\item ...
\end{itemize}

\T{H h} $\eta$\begin{itemize}
	\item H %磁场辅助场 很多作者将H而不是B,称做“磁场”。那么他们不得不为B发明一个新名字:“流密度”,或者“磁感应强度”(这个名字在电动力学中已经有至少两个其他的意义)。
\end{itemize}

\T{I i} $\iota$\begin{itemize}
	\item $I = - log_a P(x)$ %信息量
	\begin{itemize}
		\item base(2) → unit(\verb`bit`)
		\item base(e) → unit(\verb`nat`)
		\item base(10) → unit(\verb`Hartley`)
	\end{itemize}
\end{itemize}

\T{J j} $\Xi$ $\xi$\begin{itemize}
	\item $\vec{J} = \partial_{a_{VT}}\vec{I} = \rho\vec{v}$ %Electronic/Electric-Current|体电流密度 % VT: ⊥
	\begin{itemize}
		\item $F_{mag} = \int(\vec{v}\cdot\vec{B})\rho\tau = \int(\vec{J}\cdot\vec{B})\tau$
		\item $\vec{J} = \partial_{l}\vec{I} = \rho\vec{v}$% 面电流密度マ
		\item $\oint_S \vec{J}\cdot d\vec{S} = \int_\gamma(\nabla \cdot \vec{J})d\tau = -\int_\gamma(\partial_t\rho)d\tau = -\partial_t q$%局部电荷守恒 aka 连续性方程
		\item $\vec{J} = \sigma \vec{E}$ %, σ 为 电导率マ
	\end{itemize}
\end{itemize}

\T{K k} $\kappa$ \begin{itemize}
	\item ...
\end{itemize}

\T{L l} $\Lambda$ $\lambda$\begin{itemize}
	\item L %电感
		where $\Phi=LI$; $U = -LI'$.
	\item $\lambda = \dfrac Q {2 \pi R}$ %电荷线密度
	\item $\lambda = \dfrac{c}{f}$% 波长, c ぞ 光速, f ぞ 波の频率
\end{itemize}

\T{M m} $\mu$\begin{itemize}
	%\item ひな系数
	\item $\vec{M}$ %力矩
\end{itemize}

\T{N n}  $\Upsilon$ $\upsilon$ $\nu$\begin{itemize}
	\item $n$ %线圈变压器匝数比
	\item $\hat{n}$ %一个面的单位法向量 %根据求法与面内两条不平行的线垂直,列方程组或用叉乘运算
\end{itemize}

\T{O o} % omicron
\begin{itemize}
	\item ...
\end{itemize}

\T{P p} $\Pi$ $\pi$ \begin{itemize}
	\item $\vec{p} = Q \cdot \vec r$ %电(偶)极距
	%\begin{itemize}
		% 方向:正电荷指向负电荷
		% 伪方向:几何中心 → 作用点 (アリナ 定义)
		% 偶极矩是矢量,遵从矢量的加法
	%\end{itemize}
	\item %电极化强度 %aka 单位体积电偶极距
\end{itemize}

\T{Q q}\begin{itemize}
	\item Quality Factor %品质因数
	\begin{itemize}
		\item $Q = \dfrac{\omega_0 L}{R} = \dfrac 1 {\omega_0 R C}$
		\item $Q_0$ %空载品质因数,不考虑 $R_S$, $R_L$
		\item $Q_L$ %考虑电源内阻和负载的品质因数
		%\item {} Q/Circuit/Resonance
	\end{itemize}
	\item Q %电量
	\begin{itemize}
		\item Solve Method: By Gauss's law; By direct integration %(电密度)
	\end{itemize}
\end{itemize}

\T{R r} (P) $\rho$ \begin{itemize}
	\item $\rho(\vec{r})$ %体电荷密度 %{} [[F/Electronic/Static_Electro_Field]]
	\item $\rho$ %电阻率
	\item $\rho_P = -\nabla\cdot\vec{P}$ %束缚电荷密度 
\end{itemize}

\T{S s} $\Sigma$ $\sigma$ $\varsigma$
\begin{itemize}
	\item $\sigma$ 面电流密度
	\item $\sigma$ 电导率
\end{itemize}

\T{T t} $\tau$\begin{itemize}
	\item ...
\end{itemize}

\T{U u} $\theta$\begin{itemize}
	\item $\theta$ Angle
	; where $cos \theta = \dfrac{\vec{A} \cdot \vec{B}}{A \cdot B}$
\end{itemize}

\T{V v} $\Omega$ $\omega$\begin{itemize}
	\item ...
\end{itemize}

\T{W w}\begin{itemize}
	\item ...
\end{itemize}

\T{X x} $\chi$\begin{itemize}
	\item ...
\end{itemize}

\T{Y y} \begin{itemize}
	\item ...
\end{itemize}

\T{Z z} $\zeta$\begin{itemize}
	\item ...
\end{itemize}



