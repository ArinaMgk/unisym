
\subsubsection{ASCIZ String}

\B{Header}: \verb`c/ustring.h`

\subsubsection{Array}

An array owns the feature of the pointer (such as participating in pointer-arithmetic). But it can be re-point as a pointer object (because it provides spaces for elements), the array exists actually a `label' in reflected Assembly except the elements information it carried. \It{Many users fight about whether (identifier of) array is a pointer. That is a meaningless question and it depends on they consider pointer an object or a feature.}

There are two strategies to set the array. Assume an array \verb|arr[A][B][C]|:
For Row-major, the adjacent `C' always reflect the constant element, also known as \verb|(*(*(arr+A)+B)+C)|.
\begin{itemize}
	\item Magice and C
	\item C++
\end{itemize}
For Column-major, the adjacent `A' always reflect the constant element, also known as \verb|(*(*(arr+C)+B)+A)|.
\begin{itemize}
	\item Fortran
\end{itemize}

\B{EXAMPLE - Assign directly by identifier without loop or function}
\begin{lstlisting}[language=C]
	typedef struct newt {char newt[5];} newt;
	int main()
	{
		newt a = {1,2,3,4,5};
		newt b = {0};
		*(newt*)&b = *(newt*)&a;
	}
\end{lstlisting}

\paragraph{for Macro Assembler} \

\B{EXAMPLE - Pointer}
\lstset{style=GlobalASMx86}
\begin{lstlisting}[language={[x86masm]Assembler}]
	arrayA byte 10h, 20h;
	pByte typedef ptr byte;
	ptr1 pByte arrayA;
	mov esi,ptr1
	mov al,[esi]
\end{lstlisting}

In the example, we assume an array \verb|10h, 20h| and define a new pointer type \verb|pByte| for it. 

Keyword \verb`near` and \verb`far` can decide the length in some cases.
\verb`near` can be used to assign 16 or 32 bits offset,
while \verb`far` can be used to assign 32 or 48 bits offset with its segment.


\subsubsection{Enumeration, Structure and Union}

\paragraph{for Macro Assembler} \

{Definition}
\begin{lstlisting}[language={[x86masm]Assembler}]
	;: union
	union_name UNION
	union_fields
	union_name ENDS
	;: a nested case
	struct_name STRUCT
	structure_fields
	UNION union_name
	union_fields
	ENDS
	struct_name ENDS
\end{lstlisting}

Except their mechanism from literal meanings, the only difference between structure and union is that, a union allows only one initialization value.

\B{EXAMPLE - MASM Structure}

{Definition}
\begin{lstlisting}[language={[x86masm]Assembler}]
	Employee STRUCT
	IdenNum  BYTE "000000000"
	LastName BYTE 30 DUP(0)
	Years    WORD 0
	Salary   DWORD 0,0,0,0
	Employee ENDS
\end{lstlisting}

{Declaration}
\begin{lstlisting}[language={[x86masm]Assembler}]
	worker Employee <>
\end{lstlisting}

Unwritten content in Angle brackets can be ignored, written content can be overwritten; Using braces also works.

{Reference}

\verb|worker.IdenNum|; \verb|(Empolyee ptr [esi]).Years|;

\subsubsection{Class}
\subimport{.}{Class}






