
%{TODO} {Mgc} 任何语言 ---> 通用 Mika 中间语言 ---> 目标文件

\subsubsection{Parsing Stage}


\subsubsection{Comment}

\paragraph{Line After $//$} \

omit the rest part of the line
Often known as "C++-style" or "single-line" comments.
(since C99)
\begin{itemize}
\item {C, C++}
\end{itemize}

\paragraph{Line After $\#$} \
\begin{itemize}
\item Shell: Bash, ...
\item Perl (Perl has no block-comment)
\item Python
\end{itemize}

\paragraph{Line After $\;$} \
\begin{itemize}
\item A/N/YASM MASM
\end{itemize}

\paragraph{Line Head Direction} \
\begin{itemize}
\item \B{REM} Windows CMD;
\item \B{TITLE} MASM (\B{END} has it but also side-effect);
\end{itemize}

\paragraph{Special Block Rule} \ \\
\B{For MASM}

\It{\B{COMMENT} + some-symbol + ... + some-symbol}.
\begin{lstlisting}[language={[x86masm]Assembler}]
comment @ there-are-comments
there-are-comments
there-are-comments @ there-are-comments
\end{lstlisting}


\paragraph{Block Between $/*$ and $*/$} \
omit the part before the next end-flag `*/`
Often known as "C-style" or "multi-line" comments.
\begin{itemize}
\item {C, C++}
\end{itemize}

\subsubsection{Sequence Point}

\paragraph{about Calling} \

\B{First for arguments but function pointer}
In the function call \verb`(*pf[f1()]) (f2(), f3() + f4())` the functions f1, f2, f3, and f4 may be called in any order. All side effects have to be completed before the function pointed to by \verb`pf[f1()]` is called.
