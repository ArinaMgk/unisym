% UTF-8 Tex . .
%`RFG27`のAlice Dosconio: Make Mika a General-Purpose-Programming-or-Marking-Language.
%`RFB14`のAlice Dosconio: Mika is from `Micro Core` or `Mecocoa Integrated Compilation Assembly`. Henceforth, Mika is the compilers collection of the Mana. `Ma` is from `Make`, na is from `Arina/Phina` and `Haruno(Haruna)`. The name may be “Mano” in the future.
%`RFC29`のAlice Dosconio: Make Mika into a ML.
%`RFX03`のAlice Dosconio: `Mediation and General-purpose Language Compolation` (Mgl), .ML is the pre-processed documentof .Mgl.
%`RFX23`のAlice Dosconio: Rename into `Magicoll`(Universal Logical Linear Language) from `Mika` and bought the domain `magicoll.org`. 
%`20240202`のAlice Dosconio: Bought the domain `magice.org`.

\labex{Magice}

\subsection{Introduction}

\textbf{alias}: \texttt{Universal Logical Linear Language};

\textbf{domain}: \href{http://magice.org}{magice.org};

\textbf{assembly source extension}: `.a';

\textbf{Magice source extension}: `.m', `.mg', `.src'; (some format conflict with this: MATLAB, Objective-C)

\textbf{header extension}: `.h', `.mgh';

\B{used-name} magicoll, mika(now a middle language).

\subsection{Component}

\textbf{compiler}: `mgc', `magic';

Special:

1. The name of registers and instructions can be identifiers if defined \verb|_KWRD_ASM| or in block `unsafe'.

\subsection{Optimizer}

Magice have 3 built-in optimizing modes:

\begin{itemize}
\item \verb|mgc -O| + level: for running efficiency;
\item \verb|mgc -Om| + level: for memory occupation;
\item \verb|mgc -Oo| + level: for binary output occupation.
\end{itemize}

\subsection{for non-binary system architectures}
%{}...
