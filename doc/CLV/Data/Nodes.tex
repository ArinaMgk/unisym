
The chain of a node contains a series of nodes. There are usually discrete so we need not allocate them at once.

\begin{center}
\begin{longtable}{c|c|c|l}%[h!]
		\caption{Comparison between Non-annulus Chain}
		\labex{tab:table_na_nodes} \\
		%\begin{tabular}{c|c|c|l} if no using longtable
		\hline \textbf{NODE} & \textbf{DNODE} & \textbf{NNODE} & \textbf{Puts}\\
		chain\_t Chain & dchain\_t Dchain T[2] & nchain\_t Nchain & {} \\
		\hline
		\endfirsthead
		\multicolumn{4}{c}%
		{{\bfseries \tablename\ \thetable{} (continued) -- {Comparison between Non-annulus Chain}}} \\
		\hline \textbf{NODE} & \textbf{DNODE} & \textbf{NNODE} & \textbf{Puts}  \\  \hline  
		\endhead
		\hline \multicolumn{4}{r}{{Continued on the next page...}} \\
		\endfoot
		\hline \hline
		\endlastfoot
		
		\multicolumn{2}{c|}{Linear: $ArrayTrait$, $IterateTrait$} & Nested & \textit{(Form)} \\ 
		\hline
		\multicolumn{3}{c}{\$::$ReheapString$} & \textit{(S.R.S.)} \\ % "{c}" means no- split
		\hline
		\multicolumn{3}{c}{\$::$GetExtnField()$} & {} \\
		\hline
		\multicolumn{3}{c|}{$\$Insert$} & {nod,off,(type),ex-field,(on-right)$\rightarrow$(*)} \\
		\hline
		\multicolumn{3}{c|}{$\$Remove$} & {nod,(left),fn-free$\rightarrow$(*next)} \\
		\hline
		\multicolumn{3}{c|}{$\$sRelease$ $\$\$Drop$} & {since,fn-free} \\
		\hline
		\multicolumn{3}{c|}{$\$HeapFreeSimple$} & {nod} \\
		\hline
		\multicolumn{3}{c|}{\$\$$Init$} & {()} \\
		\hline
		\multicolumn{3}{c|}{\$\$::$New$ \$$New$} & {(self) or ()$\rightarrow$(*)} \\
		\hline
		\multicolumn{3}{c|}{\$\$::$Count$ \$$Count$} & {count} \\
		\hline
		\multicolumn{3}{c|}{\$\$(::)$Append$} & {(self),off,left,nod[=0 or ,type]$\rightarrow$(*new)} \\
		\hline
		\multicolumn{3}{c|}{\$\$(::)$LocateNode$} & {$\rightarrow$(*) $\|$ {index};{off,fn\_cmp=0}} \\
		\hline
		\multicolumn{3}{c|}{\$\$::$Root$ and \$\$::$RootRef$} & {$\rightarrow$(*) or (\&*)} \\
		\hline
		\multicolumn{2}{c|}{\$\$::$<<$, \$\$::$[]$} & {\$\$::$<<$, \$::$[]$} & {} \\
		\hline
		{\$\$:$getLeft$} & \multicolumn{2}{c|}{\$:$getLeft$} & {} \\
		\hline
		\multicolumn{2}{c|}{\$\$::$Last$} & {\$::$Youngest$} & {(*)} \\
		\hline
		\multicolumn{2}{c|}{\$\$::$Head$} & {\$::$Head$} & {(*)} \\
		\hline
		\multicolumn{2}{c|}{\$\$::$Tail$} & {\$::$Tail$} & {(*)} \\
		\hline
		\multicolumn{2}{c|}{\$\$::$Sorted$ \$\$$Sort$} & {/} & {(self),fn-cmp} \\
		\hline
		{} & \multicolumn{2}{c|}{\$::$GetTnodeField$} & {} \\
		\hline
\end{longtable}
\end{center}

a `\$' stands for a node, such as `Node', `Dnode' or `Nnode', while double dollars `\$\$' stands for the respective chain, such as `Chain', `Dchain' or `Nchain'.

`::' is for the class of \textit{C++} but \textit{C Programming Language}.

\subsubsection{Node Chain : Single-Field Chain}

\paragraph{Structure} \

\lstset{style=GlobalC}
\begin{lstlisting}[language=C]
	typedef struct Node {
		struct Node* next;
		union {
			const char* addr;
			pureptr_t offs;
		};
	} Node; // measures stdint[2]
\end{lstlisting}

\lstset{style=GlobalC}
\begin{lstlisting}[language=C]
#define _MACRO_CHAIN_MEMBER \
Node* root_node;\
Node* last_node;\
struct {\
	Node* midl_node;\
} fastab;\
stduint node_count;\
stduint extn_field;\
struct {\
	bool been_sorted /* `need_sort` as para of Append */;\
} state;

// C...
typedef struct NodeChain_t {
	_MACRO_CHAIN_MEMBER
	_tofree_ft func_free;
	_tocomp_ft func_comp;
} chain_t;

// C++...
class Chain : public ArrayTrait, public IterateTrait {
protected: _MACRO_CHAIN_MEMBER /*...*/
public:
	_tofree_ft func_free;
	Chain(bool defa_free = false);
	~Chain();
	/*...*/
}

\end{lstlisting}
% Node
Trait:  \verb|ArrayTrait| \verb|IterateTrait|

\paragraph{Common Method} \

In this part, the common field of the Nona-Chain will be explained.



\paragraph{Method} \


\textbf{\$\$::SortByInsertion} Sort, but by the Insertion method, which is a unique version for single linked node chain.

\textbf{\$::ReheapString$(str)$}

Treat the address of the node as a ASCIZ so re-heap it, which depends on the `$StrHeap()$'. 


\textbf{\$::GetExtnField$()$}

Get the pointer, to the end of the basic part of the node, which is the beginning of the extended part if it exists.


\textbf{\$Insert$(nod,off,(typ),exf,onr)\rightarrow*$}

Insert a node on the left or right on the node \textit{nod}. If \textit{nod} is null, this will return a pointer to the independent node. \textit{typ} exists for dnode and nnode for their owning. \textit{exf} defines the length of the extended part. \textit{onr} decides the direction, if \textit{nod} is not null.


\textbf{\$Remove$(nod,(lef),fn-free)\rightarrow*nex$} 

Remove the node no matter if in a chain. \textit{lef} is only for Node for it does not link its left node. The returns the pointer to the next node for iterating.


\textbf{\$sRelease \$\$Drop$(since,fn-free)$}

Remove the node and what on the right of it.


\textbf{\$HeapFreeSimple}

A simple releasing function. If we keep the member $func_free$ null, nothing will be done for the content of the node context. If $func_free$ points to this function, the `offs' aka `addr' will be `free'.


\textbf{\$\$Init}

Initialize the chain.


\textbf{\$\$::New$()$ or \$New$(self)$}

Create a new node context and return its pointer. The context keep the information for extended part.


\textbf{\$\$::Count$()$ or \$Count$(self)$}

Count how many nodes in the chain.


\textbf{\$\$(::)Append$(self),off,lef,nod\rightarrow*$}

Similar to `Insert' but this is for the chain. \textit{lef} decides the left ot the right when \textit{nod} is not null, the front or the tail when null.


\textbf{\$\$(::)LocateNode ...}

Unlike `Locate', always return the pointer to the node, which may be null if found not.


\textbf{\$\$::Root and \$\$::RootRef}

Return the root node of the chain. `Ref' is the reflect but reference, for Reference sometimes acts as a kind of a type in the Q'RS (Embedded Electronics) area.


\textbf{\$\$::<< and \$(\$)::[]}

Symbol style expression. ....


\textbf{\$(\$)::getLeft} 

Get the left node of the node in the chain, for `node', while get the left node of the node, no matter if it is in the chain.


\textbf{\$\$::Last or \$:Youngest} 

`Youngest' is for Nnode. Attention, Last is for the \textbf{leftest}, but youngest is for the \textbf{rightest} of the generation of the root generation.


\textbf{\$\$::Head or \$:Head} 

This is for the current generation.


\textbf{\$\$::Tail or \$:Tail} 

This is also for the current generation.


\textbf{\$\$::Sorted or \$\$Sort} 

Depends on the make interface for `Sort' trait.


\textbf{\$::GetTnodeField} 

For Dnode-compatible node. The design is for text parsing. The more to do the history commit of the git depot.



\subsubsection{Dnode Chain : Double-Field-and-Linked Chain}

\paragraph{Structure} \

\lstset{style=GlobalC}
\begin{lstlisting}[language=C]
typedef struct Dnode {
	struct Dnode* next;
	union { char* addr; pureptr_t offs; };
	struct Dnode* left;
	union { stduint type; stduint lens; };
} Dnode; // measures stdint[4]
\end{lstlisting}

\lstset{style=GlobalC}
\begin{lstlisting}[language=C]
#define _MACRO_DCHAIN_MEMBER \
Dnode* root_node;\
Dnode* last_node;\
struct {\
	Dnode* midl_node;\
} fastab;\
stduint node_count;\
stduint extn_field;\
struct {\
	bool been_sorted /* `need_sort` as para of Append */;\
} state;
	
// C...
typedef struct DnodeChain_t {
	_MACRO_DCHAIN_MEMBER
	_tofree_ft func_free;
	_tocomp_ft func_comp;
} dchain_t;
typedef struct Tnode {
	Dnode;
	TnodeField;
} Tnode;// Magice Style
	
// C++...
class Dchain : public ArrayTrait, public IterateTrait {
protected: _MACRO_DCHAIN_MEMBER /*...*/
public:
	_tofree_ft func_free;
	Dchain(); ~Dchain();
	/*...*/
}
	
\end{lstlisting}

% Dnode
Trait:  \verb|ArrayTrait| \verb|IterateTrait|

\paragraph{Method} \

\textbf{\$\$::SortByInsertion}

\textbf{DnodeRewind}

\textbf{\$\$::Reverse}

\textbf{\$\$::Match}

\textbf{\$\$::AppendMatch}

\textbf{\$\$::Reset}

\textbf{TokenParseUnit}

\subsubsection{Mnode Chain : Mapping Chain}

\subsubsection{Bnode : Binary Tree}

...

\subsubsection{Nnode Chain : Nested Chain}

\paragraph{Structure} \

\lstset{style=GlobalC}
\begin{lstlisting}[language=C]
typedef struct Nnode
{
	// Dnode
	struct Nnode* next;
	union { char* addr; void* offs; };
	union { struct Nnode* left, * pare; };
	union { stduint type; stduint lens; };
	//
	struct Nnode* subf;// sub-first item
} Nnode, nnode;
\end{lstlisting}

\lstset{style=GlobalC}
\begin{lstlisting}[language=C]
// C...
typedef struct NnodeChain_t {
	Nnode* root_node;
	stduint extn_field;
	_tofree_ft func_free;
	_tocomp_ft func_comp;
} nchain_t, Nchain;

// C++...
class Nchain {
protected:
	Nnode* root_node;
public:
	_tofree_ft func_free;
	stduint extn_field;
	//
	Nchain(bool defa_free = false); ~Nchain();
};
	
\end{lstlisting}

% Nnode
Trait: None

\paragraph{Method} \


\textbf{\$::isEldest, Nnode\_isEldest}

\textbf{NnodeEldest}
Get eldest node of parallel generation of the node.

\textbf{\$\$::DivideSymbols}

NestedParseUnit, TokenOperator, TokenOperatorGroup ...


\textbf{\$\$::Receive}

% Nnode* Receive(Nnode* insnod, DnodeChain* dnod, bool onleft = false);

\paragraph{Template} \

\lstset{style=GlobalCxx}
\begin{lstlisting}[language=C++]
// Heap String Dchain, @dosconio at 20250115
Dchain chn(DnodeHeapFreeSimple);
chn.AppendHeapstr((char*)"a");
chn.AppendHeapstr((char*)"b");
fori(i, chn) {
	forhead(i, Dnode);
	Console.OutFormat("%s\n", i.addr);
}
// Readonly String Dchain, @dosconio at 20250115
Dchain chn;
chn.Append("a");
chn.Append("b");
fori(i, chn) {
	forhead(i, Dnode);
	Console.OutFormat("%s\n", i.addr);
}
//
/*[Host Output]
* a
* b
*/
\end{lstlisting}

\subsubsection{Pnode Chain : Pool}

\subsubsection{Gnode Chain : Graph}

%\subparagraph{aaaa}
%\subsubparagraph {aaaaa}
%\subsubsubparagraph {bbbbb}

