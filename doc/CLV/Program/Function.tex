
We usually consider procedure or sub-routine only exist in Assembly. % 过程
Someone call the function returning nothing procedure. %函数

\subsubsection{Calling Statement}

Assume \verb`f` is a pointer to some function, or from an entity \verb`void f() {...}`, then \verb`f(...)` or \verb`(*f)(...)` can call the function, where \verb`...` stands for arguments actually. Even none-* or multi-* can call it, like \verb`(*****f)()`. % expi by dscn, 20240905

\subsubsection{Parameter and Argument}

\paragraph{For C or C++}\

Because pointers are in a size, it is OK to write \verb`f(int arr[1][2]) {...}` as \verb`f(int arr[][2]) { ... }` or \verb`f(int (*arr)[2]) { ... }`.
For multi arguments, the evaluation order (before calling) is decided by the compiler.

When calling the function, the item filled in the place of the respective parameter is called `argument'. Hence parameter is {formal argument}.

The form of parameter passing: reference, stack(of caller or callee), register.

The \B{storage-class specifier} for a parameter declaration is only \verb`register`.

\subsubsection{Calling Convention \mbox{-} Prologue and Epilogue }\labex{Calling}

\B{Header}: auto-inc-by stdinc.

The registers that need to be preserved before calling a function and stored after its returning is \B{Function Calling Context}. The saving subject is whether caller or callee.

\B{Consideration}
\begin{itemize}
	\item Registers need to be preserved
	\item Passing Arguments by pointer(reference), stack or registers
	\item Passing Order
	\item Returning Method
	\item Recover-register Method
\end{itemize}

\paragraph{STDCALL} \

The passing order of arguments is from right to left. The stack will be popped before returning.

\paragraph{CDECL: C Declaration} \

The passing order of arguments is from right to left. The stack will be recovered by caller. Thus {IP}, {CS}, {EFLAG} and others will not be broken.

The first argument is located closest to the top of the stack, according to which greedy arguments can be used more easily.

\paragraph{FASTCALL} \

% -------------------

\B{Register BP for Entering and Leaving Function}

%{} for functions that have arguments or local variables, ...

\B{Allocation of Stack}

This is for the parameters and locale variables. Whether the arguments are aligned with bytes depends on the compiler. For Visual C++ in Win32, it usually with multi-bytes (a \verb`dword`), while it may be  only one for GCC. Try like this if we want to assure:

\lstset{style=GlobalC}
\begin{lstlisting}[language=C]
	// ASCII C TAB4 CRLF
	// Docutitle: (Demo) 
	// Codifiers: @dosconio
	// Attribute: Endian(Little)(for %x occuper)
	#include <stdio.h>
	int main() {
		char a, b;
		printf("%x %x", &a, &b);
	}
\end{lstlisting}

\subsubsection{Inline Function for $C/C++$}

This appeared when the keyword \refer{keyword.inline} appeared.
In a hosted environment, inline cannot specify function \It{main}.

\B{EXAMPLE - Avoid side-effect}
\lstset{style=GlobalC}
\begin{lstlisting}[language=C]
inline MIN(int a, int b){if (a < b)return a; else return b;}
#define min(x,y)((x)<(y)?(x):(y))
\end{lstlisting}
Assume \verb|a| assigned "1". Then
\verb`min(++a,b);` makes \verb|a| equals 3, while
\verb`MIN(++a,b);` makes \verb|a| equals 2.

