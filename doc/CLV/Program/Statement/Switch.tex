\B{for $Magice, C,C++$}

The case labels decide the entry of the evaluation put in the parens of switch. But it is false for below.

\begin{lstlisting}[language=C]
switch (val) {
	case 'A': {} break;
	case 'B': {} break;
	// ...
	default: break;
}
\end{lstlisting}


\B{: Fall-through} 
It is a fall-through if the running path goes over another label but jump before it, after program pointer has entered a switch block from a label.
For example, if the \verb|break;| in the line \verb|'B'| is removes, the val \verb|'A'| will cause a fall-through.

\B{for $C\#$}

Based on C++.
Besides,

(1) the cases could be available labels, e.g. \verb`goto case 'A';`;

(2) \B{Inborn-String}: because the string is the inside type, it could be used directly, e.g. \verb`case 'phina':`;

\B{for $MATLAB$}
\begin{lstlisting}[language=MATLAB]
	switch val_src
	case val_0
		% ...
	case val_1
		% ...
	otherwise
		% ...
	end
\end{lstlisting}


\B{for $Verilog$}
\begin{lstlisting}[language=Verilog]
	case (data[3:0]) // assume using bits 0~3 of data
	4'h0: begin
		//...
	end
	4'h1, 4'h2: begin
		//...
	end
	default: begin
		//...
	end
	endcase
\end{lstlisting}



