\textbf{\$::ReheapString$(str)$}

Treat the address of the node as a ASCIZ so re-heap it, which depends on the `$StrHeap()$'. 


\textbf{\$::GetExtnField$()$}

Get the pointer, to the end of the basic part of the node, which is the beginning of the extended part if it exists.


\textbf{\$Insert$(nod,off,(typ),exf,onr)\rightarrow*$}

Insert a node on the left or right on the node \textit{nod}. If \textit{nod} is null, this will return a pointer to the independent node. \textit{typ} exists for dnode and nnode for their owning. \textit{exf} defines the length of the extended part. \textit{onr} decides the direction, if \textit{nod} is not null.


\textbf{\$Remove$(nod,(lef),fn-free)\rightarrow*nex$} 

Remove the node no matter if in a chain. \textit{lef} is only for Node for it does not link its left node. The returns the pointer to the next node for iterating.


\textbf{\$sRelease \$\$Drop$(since,fn-free)$}

Remove the node and what on the right of it.


\textbf{\$HeapFreeSimple}

A simple releasing function. If we keep the member $func_free$ null, nothing will be done for the content of the node context. If $func_free$ points to this function, the `offs' aka `addr' will be `free'.


\textbf{\$\$Init}

Initialize the chain.


\textbf{\$\$::New$()$ or \$New$(self)$}

Create a new node context and return its pointer. The context keep the information for extended part.


\textbf{\$\$::Count$()$ or \$Count$(self)$}

Count how many nodes in the chain.


\textbf{\$\$(::)Append$(self),off,lef,nod\rightarrow*$}

Similar to `Insert' but this is for the chain. \textit{lef} decides the left ot the right when \textit{nod} is not null, the front or the tail when null.


\textbf{\$\$(::)LocateNode ...}

Unlike `Locate', always return the pointer to the node, which may be null if found not.


\textbf{\$\$::Root and \$\$::RootRef}

Return the root node of the chain. `Ref' is the reflect but reference, for Reference sometimes acts as a kind of a type in the Q'RS (Embedded Electronics) area.


\textbf{\$\$::<< and \$(\$)::[]}

Symbol style expression. ....


\textbf{\$(\$)::getLeft} 

Get the left node of the node in the chain, for `node', while get the left node of the node, no matter if it is in the chain.


\textbf{\$\$::Last or \$:Youngest} 

`Youngest' is for Nnode. Attention, Last is for the \textbf{leftest}, but youngest is for the \textbf{rightest} of the generation of the root generation.


\textbf{\$\$::Head or \$:Head} 

This is for the current generation.


\textbf{\$\$::Tail or \$:Tail} 

This is also for the current generation.


\textbf{\$\$::Sorted or \$\$Sort} 

Depends on the make interface for `Sort' trait.


\textbf{\$::GetTnodeField} 

For Dnode-compatible node. The design is for text parsing. The more to do the history commit of the git depot.

